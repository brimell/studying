\textbf{Question 1}

At the end of a meeting all participants shake hands with each other. 28 handshakes were exchanged. How many people were at the meeting?

A. 7

B. 8

C. 9

D. 28

E. 56


\textbf{Question 2}

Find the area enclosed between the curve y = 3 + 4x – 2x2and the line y = 3.

A. 43/6

B. 23/3

C. 23/6

D. 8/3

E. 4/3


\textbf{Question 3}

A function which exhibits the following property is said to be linear:\(ƒ(\alpha x+\beta y)=\alphaƒ(x)+\beta ƒ(y)\)

A. \(ƒ(x)=x^2\)

B. \(ƒ(x)=\frac{1}{x}\)

C. \(ƒ(x)=3x\)

D. \(ƒ(x)=4x+7\)

E. \(ƒ(x)=e^x\)


\textbf{Question 4}

Consider the following function:\((9\cos^2x+6\sin x+2)^2\)

A. \(\frac{3\pi}{2}\)

B. \(\frac{-\pi}{2}\)

C. \(2\pi\)

D. \(\frac{2\pi}{3}\)

E. \(\arcsin(\frac{1}{3})\)


\textbf{Question 5}

Two tangents are drawn from the point (0 , -1) to the curve y = x2. What is the area enclosed between these tangents and the x-axis?

A. 32/3

B. 1

C. 4

D. 16/3

E. 1/2


\textbf{Question 6}

Two runners, Maria and Naeem, are racing on different circular tracks - with radii 30ft and 20ft respectively.

A. \(\frac{\pi}{3v}\)

B. \(\frac{2\pi}{3}\)

C. \(\frac{2v\pi}{5}\)

D. \(\frac{40\pi}{3v}\)

E. \(\frac{10\pi}{v}\)


\textbf{Question 7}

The following is an attempted proof of the conjecture:If\(ab\)isevenfor some integers a,b and\(a\)isodd, then\(b\)must beeven.As\(ab\)is even, it may be expressed as\(ab=2n\)As\(a\)is odd, it may be expressed as\(a = 2k + 1\).\(\Rightarrow ab=(2k+1)b\)\(\Rightarrow2n=2kb+b\)\(\Rightarrow2n-2kb=2(n-kb)=b\)Therefore, as\(b\)is a multiple of\(2\), it is even.

A. The proof is correct.

B. The proof is incorrect with an error in line 1.

C. The proof is incorrect with an error in line 2.

D. The proof is incorrect with an error in line 3.

E. The proof is incorrect with an error in line 6.


\textbf{Question 8}

The distance between two points in space P(x,-1,-1) and Q(3,-3,1) is 3. Find the possible values for x.

A. 1; 2

B. -2; -3

C. 2; 3

D. 2; 4

E. -2; -4


\textbf{Question 9}

The graph of\(y=\cos(3x+\frac{2\pi}{3})\)has a line of symmetry equal to the vertical line\(x=\alpha\).

A. \(\frac{4\pi}{7}\)

B. \(\frac{\pi}{3}\)

C. \(\frac{\pi}{9}\)

D. \(\frac{2\pi}{3}\)

E. \(\frac{2\pi}{7}\)


\textbf{Question 10}

If\({log}_b5=a,\ {log}_b2.5=c\ and{\ 5}^x=2.5\ \)then x is?

A. ac

B. \(c\over a\)

C. a+c

D. c-a

E. \(a^c\)


\textbf{Question 11}

.

A. \(n=\sqrt5m^{\frac{a-2}{4}}\)

B. \(n=2\sqrt5m^{\frac{a-1}{2}}\)

C. \(n=\sqrt{10}m^{\frac{5a-1}{2}}\)

D. \(n=2\sqrt{11}m^{\frac{a-1}{4}}\)

E. \(n=\sqrt{10}m^{\frac{a-1}{2}}\)


\textbf{Question 12}

At an arbitrary point\((p, q)\), the curve\(y=12-5x^2\)is normal to the line\(y=4-kx\)where\(k\)is a positive constant.

A. \(k=\sqrt{\frac{1}{200}}\)

B. \(k=\frac{1}{200}\)

C. \(k=\frac{1}{158}\)

D. \(k=\sqrt{\frac{1}{158}}\)

E. \(k=\sqrt{\frac{1}{160}}\)


\textbf{Question 13}

Consider a sequence,\(t_n\)which is defined iteratively as follows:\(t_1=1\)\(t_n=\displaystyle\int^1_0nt_{n-1}(u^3+u)\space du\)

A. \((\frac{3}{2})^{77}\)

B. \((\frac{3}{2})^{77}-1\)

C. \((\frac{5}{2})^{77}-77!\)

D. \((\frac{3}{4})^{76}77!\)

E. \(77(\frac{3}{4})^{77}\)


\textbf{Question 14}

Below is a diagram of a circle with radius\(r\). The two circular wedges on the left side of the circle also have radii of size r.

A. \(\frac{1}{6}r^2(\sqrt3\pi)\)

B. \(\frac{1}{6}r^2(2\sqrt3)\)

C. \(\frac{1}{2}r^2(3\sqrt\pi-1)\)

D. \(\frac{1}{6}r^2(3\sqrt3-\pi)\)

E. \(\frac{1}{2}r^2(\sqrt{3\pi}-2)\)


\textbf{Question 15}

A thin circular sheet is dipped in water so that the ratio x:y is 1:3.x is above water. Find the fraction of the circle that is submerged.

A. \(\frac{1}{2\pi}-\frac{2}{3}\)

B. \(3\over4\)

C. \(\frac{1}{2}+\pi\)

D. \(\frac{\sqrt3}{2\pi}+\frac{2}{3}\)

E. \(2 \over 3\)\(+ \)\(\sqrt 3 \over 4 \pi\)


\textbf{Question 16}

The hypotenuse of a right-angled triangle is used to form a similar triangle. This process is repeated. How many times is this operation carried out until there exists a triangle with area greater than 100 cm

A. 2

B. 7

C. 8

D. 11

E. 14


\textbf{Question 17}

Consider the following equation:\(\bigg(\frac{1}{\pi}x+\frac{1}{2}\bigg)\tan x=(1-\cos^2x)^{1/2}\)

A. No solutions.

B. \(1\)

C. \(2\)

D. \(3\)

E. An even number


\textbf{Question 18}

.

A. \(1\le x<3\)

B. \(-3\le x<-1\)

C. \(-1

D. \(-1\le x<3\)

E. \(0\le x\le3\)


\textbf{Question 19}

A group of three friends, Ruby, Sally and Tim, want to share their hal- loween candy.They distribute them such that Ruby has a ratio of two sevenths of the other two combined and Tim has five-ninths of the other two combined.

A. \(96\)

B. \(108\)

C. \(116\)

D. \(126\)

E. \(184\)


\textbf{Question 20}

What is the ratio of the total surface area of an sphere to the total surface area of a hemisphere?

A. 1:3

B. 2:3

C. 4:3

D. 3:4

E. 3:2


\textbf{Question 21}

.

A. \(\frac{1}{\sqrt{y^3}}\)

B. \(y^{-1/2}\)

C. \(\frac{1}{1+y^{1/2}}\)

D. \(\frac{1}{y}\)

E. \(\sqrt{y^3}\)


\textbf{Question 22}

Consider the function below:\(p(x)=\frac{2}{3}x^3+\frac{3}{2}x^2-2x+15\)

A. \(\frac{1}{2}\le x<\infty\)

B. \(-\infty

C. \(x<-2,x>\frac{1}{2}\)

D. \(-2

E. \(\infty\le x<\frac{1}{2}\)


\textbf{Question 23}

Express b as a function of a, where\(a={log}_32,\ b={log}_{12}18\).

A. \(\frac{a+1}{a+2}\)

B. \(\frac{a+2}{2a+1}\)

C. \(\frac{3a+1}{2a+1}\)

D. \(\frac{a-2}{2a+1}\)

E. \(\frac{a-3}{3a+\sqrt2}\)


\textbf{Question 24}

Last week, police ticketed 13 men travelling 18 miles per hour over the speed limit and 8 women traveling 14 miles per hour over the speed limit. What was the mean speed over the limit of all 21 drivers?

A. 16 miles per hour

B. 16.5 miles per hour

C. 17 miles per hour

D. 15 miles per hour

E. 12 miles per hour


\textbf{Question 25}

If a, b and c are real numbers and if\(a^5b^3c^8=\frac{9a^3c^8}{b^{-3}},\ \)then a could equal?

A. \(\frac{1}{9}b^6\)

B. \(1\over3\)

C. 9

D. 3

E. \({9b}^6\)


\textbf{Question 26}

As\(x\)varies over all real numbers, find the number of solutions of the following function.

A. No solutions.

B. 1

C. 2

D. 8

E. Infinitely many.


\textbf{Question 27}

As shown below, there is square piece which has a semi-circular cut-out. The area of this entire piece is 20 units.

A. \(x=2\sqrt\frac{5}{8-3\pi}\)

B. \(x=4\sqrt\frac{10}{8-\pi}\)

C. \(x=2\sqrt\frac{5}{4-\pi}\)

D. \(x=4\sqrt\frac{11}{8-3\pi}\)

E. \(x=\sqrt\frac{5}{2-\pi}\)


\textbf{Question 28}

Determine\(f\)in terms of\(x\), given the following equation:\(\frac{1}{x}\frac{df}{dx}=\frac{x^2+7x+2}{x^5}\)

A. \(ƒ(x)=-x^{-1}-\frac{7}{2}x^{-2}-\frac{2}{3}x^{-3}+\frac{49}{6}\)

B. \(ƒ(x)=-3x^{-1}-\frac{5}{2}x^{-2}-\frac{2}{3}x^{-3}+\frac{19}{6}\)

C. \(ƒ(x)=-x^{-1}-\frac{7}{2}x^{-2}-\frac{5}{3}x^{-3}+\frac{31}{6}\)

D. \(ƒ(x)=-x^{-1}-\frac{3}{2}x^{-2}-\frac{2}{3}x^{-3}+-\frac{7}{3}\)

E. \(ƒ(x)=-x^{-1}-\frac{11}{2}x^{-2}-\frac{2}{3}x^{-3}+\frac{29}{6}\)


\textbf{Question 29}

Consider the following quadratic function:\(ƒ(x,\beta)=x^2-2\beta x+1\)

A. \(\frac{1}{8}\)

B. \(0\)

C. \(\frac{1}{4}\)

D. \(\frac{1}{\sqrt2}\)

E. \(\frac{1}{12}\)


\textbf{Question 30}

8000 cuboids of dimensions 2x, x and x are melted and formed into a similar cuboid. Find the ratio of the total surface area of the individual cuboids to that of the new cuboid.

A. 8000:1

B. 20:1

C. 9:20

D. 200:9

E. 46:2


\textbf{Question 31}

What is the probability that a prime number is less than 7, given that it is less than 13?

A. \(\frac{1}{3}\)

B. \(\frac{2}{5}\)

C. \(\frac{1}{2}\)

D. \(\frac{3}{5}\)

E. \(\frac{3}{4}\)


\textbf{Question 32}

Evaluate the area of the enclosed region as defined by the following in equalities:

A. \(\frac{108}{7}\)

B. \(\frac{121}{4}\)

C. \(\frac{33}{8}\)

D. \(\frac{63}{4}\)

E. \(\frac{13}{12}\)


\textbf{Question 33}

Solve the two equations:\(5^x+5^{-x}=2 \)and\({log}_{10}y+{log}_{10}\left(9-2y\right)=1.\)What is the sum x+y, choosing the smaller solution for y?

A. 2

B. \(5\over2\)

C. 1

D. 3

E. 5


\textbf{Question 34}

\(\int_{0}^{2}{{(x-\sqrt x)}^2dx}.\)Calculate the following integral:

A. \(\frac{14}{3}-\frac{16}{5}\sqrt2\)

B. \(\frac{14}{3}+\frac{16}{5}\sqrt2\)

C. \(4-\frac{16}{5}\sqrt2\)

D. \(\frac{14}{3}\)

E. None of the above


\textbf{Question 35}

.

A. \(x=0\)

B. \(x=2e\)

C. \(x=e^{1/2}\)

D. \(x=e^{-1}\)

E. \(x=e^{-1/2}\)


